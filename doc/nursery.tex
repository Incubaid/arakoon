%\section{Scaling arakoon}
%We want to be able to use arakoon for increasingly large key-value spaces. 
%For a single arakoon cluster the capacity is limited by the size of a single disk. So it is only natural to allow different arakoon clusters to team up. 
%A nursery\footnote{after a \emph{a nursery of raccoons}} provides a semi-unified view on a set of arakoon clusters. 
%Each cluster is uniquely responsible for a prefix range. 
%\subsection{Limitations}
%\subsubsection{impact on sequences}
%Sequences are multiple updates that are done atomically. 
%Since atomicity can only be achieved inside 1 cluster, this means that all keys for a sequence need to share the same prefix.
%\subsubsection{impact on ranges}
%Every cluster is responsible for a specific range. 
%As client range query will only be served by a single cluster, it means that only ranges that are subranges of cluster ranges can be served.
%\subsection{Migrations}
%Once a cluster is filled, one needs to be able to split it, or move part of its range elsewhere. This process is called migration.
%Each cluster has a \emph{public} range $[k_b,k_e)$ it serves. 
%Each cluster also has a \emp{private} range it contains. 
%As such, migrating a part of a cluster's range to another cluster becomes feasible. If we're moving keys away from a \emph{source} cluster to a \emph{target} cluster, we
%\begin{itemize}
%\item{} step 1
%\item{} step 2
%\item{} step 3
%\item{} step 4
%\end{itemize}
%\section{Limitations to Tokyo Cabinet}
%Tokyo Cabinet will not cope with values of 1MB; event 10KB gives problems if yo%u don't tweak the parameters. In essence, allowing both 1B and 1MB in the same %Btree will explode.
